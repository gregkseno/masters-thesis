\documentclass{article}
\usepackage{arxiv}

\usepackage{amsmath}
\usepackage[linesnumbered,ruled,vlined]{algorithm2e}
\usepackage[utf8]{inputenc}
\usepackage[english, russian]{babel}
\usepackage[T1]{fontenc}
\usepackage{url}
\usepackage{dsfont}
\usepackage{booktabs}
\usepackage{amsfonts}
\usepackage{nicefrac}
\usepackage{microtype}
\usepackage{lipsum}
\usepackage{graphicx}
\usepackage{natbib}
\usepackage{doi}

\SetKwInput{KwInput}{Input}                % Set the Input
\SetKwInput{KwOutput}{Output}              % set the Output



\title{The Schrödinger Bridge Problem in GAN}
% \thanks{Use footnote for providing further
% 		information about author (webpage, alternative
% 		address)---\emph{not} for acknowledging funding agencies.} 
\author{ Gregory S.~Ksenofontov\\
	Department of Intelligent Systems\\
	Moscow Institute of Physics and Technology (National Research University)\\
	1 “А” Kerchenskaya st., Moscow, 117303, Russian Federation \\
	\texttt{ksenofontov.gs@phystech.edu} \\
	% \And
	% Elias D.~Striatum \\
	% Department of Electrical Engineering\\
	% Mount-Sheikh University\\
	% Santa Narimana, Levand \\
	% \texttt{stariate@ee.mount-sheikh.edu} \\
	%% \AND
	%% Coauthor \\
	%% Affiliation \\
	%% Address \\
	%% \texttt{email} \\
	%% \And
	%% Coauthor \\
	%% Affiliation \\
	%% Address \\
	%% \texttt{email} \\
	%% \And
	%% Coauthor \\
	%% Affiliation \\
	%% Address \\
	%% \texttt{email} \\
}
\date{}

\renewcommand{\shorttitle}{\textit{arXiv} Template}

%%% Add PDF metadata to help others organize their library
%%% Once the PDF is generated, you can check the metadata with
%%% $ pdfinfo template.pdf
\hypersetup{
    pdftitle={Master's Thesis},
    pdfsubject={q-bio.NC, q-bio.QM},
    pdfauthor={ Gregory S.~Ksenofontov},
    pdfkeywords={Schrödinger Bridge Problem, GAN, domain adaptation},
}

\begin{document}
\maketitle

\begin{abstract}
	\lipsum[1]
\end{abstract}


\keywords{Schrödinger Bridge Problem \and GAN \and Domain Adaptation}

\section{Introduction}
\subsection{What is Schrödinger Bridge Problem?}
Let it is given two  distributions $\pi_0(x)$ and $\pi_1(y)$ (where at time 0 and 1), that describes two data distributions of  functions  $x, y\in\mathcal{C}([0,1], \mathbb{R}^d)$ at $t=0$ and time $t=1$, respectively. Also, consider:
$$\pi_1(y) \neq \int p(y|x)\pi_0(x)dx,$$
where $p(y_1|x_0)$ -- the transition density (i.e. the probability of transitioning from $x$ at time $0$ to $y$ at time $1$) under the Brownian motion. In other words, $\pi_1(y)$ differs significantly from the distribution predicted by Brownian motion (т.е. наше априорный Винеровский процесс не ограничен $\pi_0(x)$ and $\pi_1(y)$). \par 
% The Schrödinger Bridge problem considers the finding of most likely stochastic process that transforms the distribution $\pi_0(x)$ at time $0$ to $\pi_1(y)$ at time $1$ and is closer in information-theoretic sense.
Основная мотивация Schrödinger Bridge Problem найти наиболее вероятный стохостический процесс между $\pi_0(x)$ and $\pi_1(y)$ относительно заданного (и не ограниченного) априорного процесса. Для этого рассмотрим некоторый набор броуновских движений (винеровских процессов) \{$x_i(t)\}^N_{i=1}$, где каждый элемент данного множества принадлежит множеству всех функций $\mathcal{C}([0,1], \mathbb{R}^d)$, т.е. $x:[0,1]\rightarrow \mathbb{R}^d$.\par 
Также зададим новое понятие Path measure. Path measure is a weak solution for given SDE, i.e. a stochastic process. Weak solution is a terminology for a solution of an SDE, that does not take into account an initial value problem and has the freedom of specifying its probability space. В дискретном случае path measure определяется слудющим образом:


\begin{equation} \label{eq1}   
    \hat{\mathbb{W}}(A)=\frac{1}{N}\sum_{i=1}^N\mathds{1}\left(x_i(t)\in A\right), A\in \mathcal{B}(\mathbb{R}^d)^{[0,1]}
\end{equation}


Однако, данное определение не гарантирует, что at time $0$ to $y$ at time $1$, распределение будет совпадать с заданным. Поэтому теперь ограничим path measure $\hat{\mathbb{W}}$:
\begin{equation} \label{eq2}
    P\left(\hat{\mathbb{W}} \in \mathcal{D}(\pi_0, \pi_1)\right)=1
\end{equation}
Т.е. вероятность того, что данный path measure принадлежит множеству всех path measures with two prescribed marginals равнялась единице. Из теоремы Санова, изветстно, что асимтотически данная вероятность стремится к следующему выражению:
\begin{equation} \label{eq3}
    P\left(\hat{\mathbb{W}} \in \mathcal{D}(\pi_0, \pi_1)\right) \xrightarrow{N\rightarrow \infty} \exp\left(-N\inf_{\mathbb{Q} \in \mathcal{D}(\pi_0, \pi_1)}KL\mathbb{(Q||W})\right)
\end{equation}

Таким образом, для того, чтобы выбполнялось условие \ref{eq2} требуется, чтобы $\mathbb{KL}$ дивергенция была минимальной.
Overall, the Schrödinger bridge provides us with a theoretically-grounded mechanism for mapping between two distribution to solve such problems like unsupervised domain adaptation and also gives the with the probability of this stochastic evolution, thus allowing us to compare two datasets/distributions which can be useful for hypothesis testing and semantic similarity
So, speaking formally, there are two approaches to problem setting, dynamic and static.
\subsection{Dynamic Schrödinger Bridge Problem}
Данная постановка задачи выходит из теоремы \ref{eq3} . In this approach is minimizing of KL divergence between two \textit{path measures} $\mathbb{Q}$ and $\mathbb{W}^\gamma$, path measures of our desired process and prior Wiener process with drift coefficient $\sqrt\gamma$.
\begin{equation}\label{dyn}
    \hat{\mathbb{Q}} = \arg\min_{\mathbb{Q}\in \mathcal{D}(\pi_0, \pi_1)} KL\mathbb{(Q||W}^\gamma)
\end{equation}

\subsection{Static Schrödinger Bridge Problem}
Let's decompose $KL$ divergence in \ref{dyn}. Firstly, lets reexpress RN-derivative conditioning on $\textbf{x}(0)=x$ and $\textbf{x}(1)=y$:
$$
\frac{d\mathbb{Q}}{d\mathbb{W}^\gamma} = \frac{q(x, y)}{p^{\mathbb{W}^\gamma}(x, y)}\frac{d\mathbb{Q}_{(0,1)}}{d\mathbb{W}^\gamma_{(0,1)}}(\cdot|x,y)
$$
Passing this expression to $\mathbb{KL}$ divergence in \ref{dyn} we get:
\begin{equation}
    \begin{split}
        KL\mathbb{(Q||W}^\gamma) = \mathbb{E_Q}\left[\log \frac{d\mathbb{Q}}{d\mathbb{W}^\gamma}\right] = \mathbb{E_Q}\left[\log \frac{q(x, y)}{p^{\mathbb{W}^\gamma}(x, y)}\frac{d\mathbb{Q}_{(0,1)}}{d\mathbb{W}^\gamma_{(0,1)}}(\cdot|x,y)\right] = \\ = \int\log \frac{q(x, y)}{p^{\mathbb{W}^\gamma}(x, y)} d\mathbb{Q} + \int\log \frac{d\mathbb{Q}_{(0,1)}}{d\mathbb{W}^\gamma_{(0,1)}}(\cdot|x,y)d\mathbb{Q}= \\ = \int q(x, y)\log \frac{q(x, y)}{p^{\mathbb{W}^\gamma}(x, y)} dxdy + \int\log q(x, y)\frac{d\mathbb{Q}_{(0,1)}}{d\mathbb{W}^\gamma_{(0,1)}}(\cdot|x,y)d\mathbb{Q}_{(0,1)} = \\ = KL\left(q(x, y) || p^{\mathbb{W}^\gamma}(x, y)\right) + \mathbb{E}_{q(x, y)}\left[KL\left(\mathbb{Q}_{(0,1)}(\cdot|x,y) || \mathbb{W}^\gamma_{(0,1)}(\cdot|x,y)\right)\right]
    \end{split}
\end{equation}
We can see that the second $KL$ divergence doesn't consider the marginal distributions, so it doesn't affect optimisation constraint. Thus, zeroing it we get  static statement:
$$\left\{ \begin{array}{c}
\hat q(x,y) = \arg\min_{q(x,y)} KL(q(x,y)||p^{\mathbb{W}^\gamma}(x,y)), \\
\pi_0(x) = \int q(x,y)dy, \\
\pi_1(y) = \int q(x,y)dx
\end{array}\right.$$
where $q(x,y)$ is a joint distribution which is closest to the Brownian-motion prior subject to marginal constraints (between PDF at time 0 and 1)
\subsection{Schrödinger System}
Applying Lagrangian on static SBP we get
\begin{equation}
    L(q, \lambda, \mu) = KL(q(x,y)||p^{\mathbb{W}^\gamma}(x,y)) + \int \lambda(x)\left(\int q(x, y)dy - \pi_0(x)\right)dx + \int \mu(y)\left(\int q(x, y)dy - \pi_1(y)\right)dy
\end{equation}
Предполагая, что $p^{\mathbb{W}^\gamma}(x,y) = p_0^{\mathbb{W}^\gamma}(x)p^{\mathbb{W}^\gamma}(y|x)$, где $p_0^{\mathbb{W}^\gamma}(x)$ может быть любым, а $p^{\mathbb{W}^\gamma}(y|x)=\mathcal{N}(y|x, \gamma I_d)$ (это выходит из свойств Винеревского процесса).  Теперь приравняем нулю $ \frac{\partial L(q, \lambda, \mu)}{\partial q(x,y)}$
\begin{equation}
    q^*(x, y) = \exp{(\ln{p^{\mathbb{W}^\gamma}(x,y)} - \lambda(x) - 1)}p^{\mathbb{W}^\gamma}(y|x)\exp{(-\mu(y))}
\end{equation}
Теперь положив, что  $\hat\phi_0(x) = \exp{(\ln{p^{\mathbb{W}^\gamma}(x,y)} - \lambda(x) - 1)}$ и $\phi_1(y) = \exp{(-\mu(y))}$ получаем
\begin{equation}\label{eq7}
    q^*(x, y) = \hat\phi_0(x)p^{\mathbb{W}^\gamma}(y|x)\phi_1(y)
\end{equation}
маргинализуя данное распределение по $x$ или $y$ получим
\begin{equation}
\begin{split}
    \pi_0(x) = \hat\phi_0(x)\phi_0(x) \\
    \pi_1(y) = \phi_1(y)\hat\phi_1(y)
\end{split}
\end{equation}
где  $\phi_0(x) = \int\phi_1(y)p^{\mathbb{W}^\gamma}(y|x)dy$, а $\hat\phi_1(y) = \int\hat\phi_0(x)p^{\mathbb{W}^\gamma}(y|x)dx$

\subsection{Schrödinger Half Bridges}
The half bridge problem consider only one of the boundaries constraint, i.e. $\mathcal{D}(\pi_0(x), \cdot)$ or $\mathcal{D}(\cdot, \pi_1(y))$. 
\subsubsection{Dynamic Half Bridges}
More formally forward half bridge is given by:
\begin{equation}
    \mathbb{Q}^*=\arg\min_{\mathbb{Q}\in\mathcal{D}(\pi_0(x), \cdot)}KL\left(\mathbb{Q}||\mathbb{W}^\gamma\right)
\end{equation}
and backward half bridge is given by:
\begin{equation}
    \mathbb{P}^*=\arg\min_{\mathbb{P}\in\mathcal{D}(\cdot, \pi_1(y))}KL\left(\mathbb{P}||\mathbb{W}^\gamma\right)
\end{equation}
And they admits following solutions, respectively 
\begin{equation}
    \mathbb{Q}^*(A_0\times A_{(0,1]})=\int_{A_0\times A_{(0,1]}}\frac{d\pi_0}{p_0^{\mathbb{W}^\gamma}}(x)d\mathbb{W}^\gamma
\end{equation}
\begin{equation}
    \mathbb{P}^*(A_{[0,1)} \times A_1)=\int_{A_{[0,1)} \times A_1}\frac{d\pi_1}{p_1^{\mathbb{W}^\gamma}}(y)d\mathbb{W}^\gamma
\end{equation}
\subsubsection{Static Half Bridges}
Static forward half bridge is given by:
\begin{equation}
\begin{split}
    q(x,y)^*=\arg\min_{q(x,y)\in\mathcal{D}(\pi_0(x), \cdot)}KL\left(q(x,y)||p^{\mathbb{W}^\gamma}(x,y)\right) \\
    s.t. \pi_0(x) = \int q(x, y)dy
\end{split}
\end{equation}
and backward half bridge is given by:
\begin{equation}
\begin{split}
    p(x,y)^*=\arg\min_{p(x,y)\in\mathcal{D}(\cdot, \pi_1(y))}KL\left(p(x,y)||p^{\mathbb{W}^\gamma}(x,y)\right) \\
    s.t. \pi_1(y) = \int q(x, y)dx
\end{split}
\end{equation}
And they admits following solutions, respectively
\begin{equation}
    q(x,y)^*=p(x,y)^{\mathbb{W}^\gamma}\frac{\pi_0(x)}{p^{\mathbb{W}^\gamma}(x)}
\end{equation}
\begin{equation}
    p(x,y)^*=p(x,y)^{\mathbb{W}^\gamma}\frac{\pi_1(y)}{p^{\mathbb{W}^\gamma}(y)}
\end{equation}

Half bridges are a significantly easier problem than full bridges. Not only do they admit “closed-form” solutions in some sense, they also allow removing constraints by incorporating them as an initial value problem
\subsection{Iteration Proportional Fitting (IPF)}
One of the oldest algorithms for solving Schrödinger Bridge Problem is Fortet's algorithm (1940).

\begin{algorithm}
\caption{Fortet's Algorithm}\label{alg:fortret}
\KwInput{$\pi_0(x)$, $\pi_1(y)$, $p(y|x)$}
\KwOutput{$\hat\phi^{(i)}_0(x)$, $\phi^{(i)}_1(y)$}
Initialize $\phi_0^{(0)}(x)$ s.t. $\phi_0^{(0)}(x)<<\pi_0(x)$\;
\While{not converged}{
    $\hat\phi_0^{(i)}(x):=\frac{\pi_0(x)}{\phi_0^{(i)}(x)}$\;
   $\hat\phi_1^{(i)}(y):=\int p(y|x)\hat\phi_0^{(i)}(x)dx$\;
    $\phi_1^{(i)}(y):=\frac{\pi_1(y)}{\hat\phi_1^{(i)}(y)}$\;
    $\hat\phi_1^{(i+1)}(x):=\int p(y|x)\phi_1^{(i)}(y)dy$\;
    $i:=i+1$\;
   }
\end{algorithm}

\section{Related works}
\subsection{Diffusion Schrodinger Bridge Matching (DSBM)}
\citet{dsbm}  proposed new method of solving SBP in dynamic way using Iterative Markovian Fitting
\subsection{Diffusion Schrodinger Bridge with Applications to Score-Based Generative Modeling}
\citet{dsb} proposed using SBP in dynamic way to build generative model


% \bibliographystyle{acm}
\bibliographystyle{unsrtnat}
\bibliography{references}

\end{document}