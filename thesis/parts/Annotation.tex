\begin{abstract}

    \begin{center}
        \large{Состязательные мосты Шрёдингера на задаче трансляции домена} \\
    \large\textit{Ксенофонтов Григорий Сергеевич} \\[1 cm]
    \end{center}    
    В магистерской диссертации рассматривается задача трансляции домена. Данная задача заключается в нахождении отображений $G$ и $K$, которые транслируют элементы из набора $X$ в $Y$ и наоборот. Передовым методом для решения этой задачи являются мосты Шрёдингера -- прямой и обратный стохастические процессы, ограниченные заданными распределениями. Их особенностью среди других методов трансляции домена, таких как CycleGAN, является дополнительное свойство оптимальности трансляции.
    
    Целью исследования является разработка нового метода нахождения мостов Шрёдингера, который решает две проблемы: необходимость в моделировании стохастического процесса, а также проклятие размерности. Для решения этих проблем, в данной работе предложен новый подход, основанный на состязательном обучении. Такой метод объединяет преимущества состязательных генеративных сетей (GAN) и мостов Шрёдингера. В данной работе проводятся эксперименты предложенного подхода на различных наборах данных, включая 2D данные и EMNIST. Полученные результаты показывают, что предложенный метод удовлетворяет всем поставленным требованиям.
    
    \vfill
    
    \begin{center}
    \textbf{Abstract} \\[1 cm]
    Adversarial Schrödinger bridges on domain translation problem
    \end{center}
    The master's thesis examines the domain translation problem. This problem is to find a mapping $G$ and $K$ that translate elements from the set $X$ to $Y$ and vice versa. The advanced method for solving this problem is Schrödinger bridges - direct and inverse stochastic processes that are limited by given distributions. Their main feature among other domain translation methods, such as CycleGAN, is the additional property of the optimality of the resulting translation.

    The goal of the study is to develop a new method for finding Schrödinger bridges that adresses two problems: the need to model a stochastic process, as well as the curse of dimensionality. To solve these problems, it is proposed a new approach based on adversarial learning. This method combines the advantages of adversarial generative networks (GANs) and Schrödinger bridges. In this work, experiments of the proposed approach are carried out on various datasets, including 2D data and EMNIST. The results obtained show that the proposed method satisfies all the stated requirements.
\end{abstract}
\newpage