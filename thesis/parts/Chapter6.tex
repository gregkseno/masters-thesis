\section{Заключение}
\label{sec:Chapter6} \index{Chapter6}

В ходе выполнения данной магистерской диссертации были достигнуты следующие результаты:
\begin{enumerate}
    \item Проведен обзор и анализ различных подходов к решению задачи мостов Шрёдингера, включая Data Driven Schrödinger Bridge, Light Schrödinger Bridge, Diffusion Schrödinger Bridge, Iterative Proportional Maximum Likelihood (IPML) и Unpaired Neural Schrödinger Bridge.
    \item Предложен метод состязательных мостов Шрёдингера, объединяющий идеи генеративных состязательных сетей (GAN) и задачу мостов Шрёдингера.
    \item Проведен эксперимент на 2D данных с использованием наборов данных кольца, луны и рулета из библиотеки sci-kit learn. Параметры набора данных были выбраны таким образом, чтобы проверить работоспособность метода и его способность к отображению при различных параметрах $\gamma$, а также проверить оптимальность отображения, что подтверждает пригодность предложенного метода для решения задачи мостов Шрёдингера.
    \item Проведено сравнение отображений CycleGAN и предложенного метода.
    \item Проведен эксперимент на данных большой размерности с использованием набора данных EMNIST, который подтверждает способность метода работать с большими и сложными наборами данных.
\end{enumerate}

В итоге, можно заключить, что задача, поставленная в данной магистерской диссертации, была решена в полной мере. Разработанный метод состязательных мостов Шрёдингера доказал свою эффективность и применимость для задач трансляции доменов. Проведенные эксперименты подтвердили теоретические выводы и продемонстрировали высокую эффективность предложенного подхода. В будущем данный метод может быть расширен и адаптирован для решения более широкого круга задач в области машинного обучения и анализа данных.

\newpage
