\section{Постановка задачи}
\label{sec:Chapter2} \index{Chapter2}

В контексте машинного обучения зачастую невозможно получить распределение данных в явной форме. Это вызывает необходимость рассмотрения альтернативных постановок задачи мостов Шрёдингера, в которых маргинальные распределения известны лишь посредством выборок, представленных в виде двух наборов данных $\{x_i\}_{i=0}^N = X \in \mathbb{R}^{N\times d}$ и $ \{y_j\}_{j=0}^M = Y\in \mathbb{R}^{M\times d}$. 

Эмпирическими мостами Шрёдингера называется такая постановка задачи мостов Шрёдингера, в которой маргинальные распределения известны только через выборки (то есть их эмпирические распределения):

\begin{equation*}
    \hat{\pi}_0(x) = \frac{1}{N} \sum_{i=1}^{N} \delta(x - x_i), \quad \hat{\pi}_T(y) = \frac{1}{M} \sum_{j=1}^{M} \delta(y - y_j),
\end{equation*}
где $\delta$ -- дельта-функция Дирака.

Элементы этих наборов принадлежат соответствующим распределениям, а именно $x_0 = x \sim \pi_0(x)$ и $x_T = y \sim \pi_T(y)$.
Проанализировав существующие подходы, можно заключить, что большинство методов, таких как Diffusion Schrödinger Bridge \cite{dsb}, Iterative Proportional Maximum Likelihood (IPML) \cite{mle-sb} и Unpaired Neural Schrödinger Bridge \cite{cycle-sb}, основываются на моделировании дискретизированного стохастического процесса, представленного марковской цепочкой:
\begin{equation*}
q(x_0, \dots, x_T) = \pi_1(x_T) \prod_{t=0}^{T-1} q_{t|t+1} (x_t | x_{t+1}).
\end{equation*}

В этих методах условное распределение $q_{t|t+1}$ параметризуется с использованием нейронных сетей. Для преобразования одного набора данных в другой в этих методах итеративно сэмплируются элементы с использованием обученного $q_{t|t+1}$, таким образом моделируя стохастический процесс. В результате таких преобразований требуется несколько шагов для переноса элемента из одного набора в другой. Несмотря на значительный прогресс в сокращении числа необходимых шагов, такие методы все еще являются вычислительно сложными задачами.

С другой стороны, методы, которые рассматривают задачу мостов Шрёдингера без моделирования процесса, такие как Data Driven Schrödinger Bridge \cite{pmlr-v139-wang21l} и Light Schrödinger Bridge \cite{lsb}, плохо справляются с данными большой размерности из-за необходимости явной параметризации и обучения условного распределения.

В связи с этим, возникает необходимость разработки подхода к решению задачи мостов Шрёдингера, который бы обладал следующими качествами:
\begin{itemize}
\item преобразование из одного распределения в другое должно осуществляться за один шаг;
\item метод должен быть эффективным при работе с данными большой размерности.
\end{itemize}

Таким образом, цель данной работы состоит в предложении решения задачи мостов Шрёдингера, которое бы удовлетворяло указанным выше требованиям. 

\newpage